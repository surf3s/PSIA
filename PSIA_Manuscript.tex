% Template for PLoS
% Version 3.5 March 2018
%
% % % % % % % % % % % % % % % % % % % % % %
%
% -- IMPORTANT NOTE
%
% This template contains comments intended
% to minimize problems and delays during our production
% process. Please follow the template instructions
% whenever possible.
%
% % % % % % % % % % % % % % % % % % % % % % %
%
% Once your paper is accepted for publication,
% PLEASE REMOVE ALL TRACKED CHANGES in this file
% and leave only the final text of your manuscript.
% PLOS recommends the use of latexdiff to track changes during review, as this will help to maintain a clean tex file.
% Visit https://www.ctan.org/pkg/latexdiff?lang=en for info or contact us at latex@plos.org.
%
%
% There are no restrictions on package use within the LaTeX files except that
% no packages listed in the template may be deleted.
%
% Please do not include colors or graphics in the text.
%
% The manuscript LaTeX source should be contained within a single file (do not use \input, \externaldocument, or similar commands).
%
% % % % % % % % % % % % % % % % % % % % % % %
%
% -- FIGURES AND TABLES
%
% Please include tables/figure captions directly after the paragraph where they are first cited in the text.
%
% DO NOT INCLUDE GRAPHICS IN YOUR MANUSCRIPT
% - Figures should be uploaded separately from your manuscript file.
% - Figures generated using LaTeX should be extracted and removed from the PDF before submission.
% - Figures containing multiple panels/subfigures must be combined into one image file before submission.
% For figure citations, please use "Fig" instead of "Figure".
% See http://journals.plos.org/plosone/s/figures for PLOS figure guidelines.
%
% Tables should be cell-based and may not contain:
% - spacing/line breaks within cells to alter layout or alignment
% - do not nest tabular environments (no tabular environments within tabular environments)
% - no graphics or colored text (cell background color/shading OK)
% See http://journals.plos.org/plosone/s/tables for table guidelines.
%
% For tables that exceed the width of the text column, use the adjustwidth environment as illustrated in the example table in text below.
%
% % % % % % % % % % % % % % % % % % % % % % % %
%
% -- EQUATIONS, MATH SYMBOLS, SUBSCRIPTS, AND SUPERSCRIPTS
%
% IMPORTANT
% Below are a few tips to help format your equations and other special characters according to our specifications. For more tips to help reduce the possibility of formatting errors during conversion, please see our LaTeX guidelines at http://journals.plos.org/plosone/s/latex
%
% For inline equations, please be sure to include all portions of an equation in the math environment.
%
% Do not include text that is not math in the math environment.
%
% Please add line breaks to long display equations when possible in order to fit size of the column.
%
% For inline equations, please do not include punctuation (commas, etc) within the math environment unless this is part of the equation.
%
% When adding superscript or subscripts outside of brackets/braces, please group using {}.
%
% Do not use \cal for caligraphic font.  Instead, use \mathcal{}
%
% % % % % % % % % % % % % % % % % % % % % % % %
%
% Please contact latex@plos.org with any questions.
%
% % % % % % % % % % % % % % % % % % % % % % % %

\documentclass[10pt,letterpaper]{article}
\usepackage[top=0.85in,left=2.75in,footskip=0.75in]{geometry}

% amsmath and amssymb packages, useful for mathematical formulas and symbols
\usepackage{amsmath,amssymb}

% Use adjustwidth environment to exceed column width (see example table in text)
\usepackage{changepage}

% Use Unicode characters when possible
\usepackage[utf8x]{inputenc}

% textcomp package and marvosym package for additional characters
\usepackage{textcomp,marvosym}

% cite package, to clean up citations in the main text. Do not remove.
% \usepackage{cite}

% Use nameref to cite supporting information files (see Supporting Information section for more info)
\usepackage{nameref,hyperref}

% line numbers
\usepackage[right]{lineno}

% ligatures disabled
\usepackage{microtype}
\DisableLigatures[f]{encoding = *, family = * }

% color can be used to apply background shading to table cells only
\usepackage[table]{xcolor}

% array package and thick rules for tables
\usepackage{array}

% create "+" rule type for thick vertical lines
\newcolumntype{+}{!{\vrule width 2pt}}

% create \thickcline for thick horizontal lines of variable length
\newlength\savedwidth
\newcommand\thickcline[1]{%
  \noalign{\global\savedwidth\arrayrulewidth\global\arrayrulewidth 2pt}%
  \cline{#1}%
  \noalign{\vskip\arrayrulewidth}%
  \noalign{\global\arrayrulewidth\savedwidth}%
}

% \thickhline command for thick horizontal lines that span the table
\newcommand\thickhline{\noalign{\global\savedwidth\arrayrulewidth\global\arrayrulewidth 2pt}%
\hline
\noalign{\global\arrayrulewidth\savedwidth}}


% Remove comment for double spacing
%\usepackage{setspace}
%\doublespacing

% Text layout
\raggedright
\setlength{\parindent}{0.5cm}
\textwidth 5.25in
\textheight 8.75in

% Bold the 'Figure #' in the caption and separate it from the title/caption with a period
% Captions will be left justified
\usepackage[aboveskip=1pt,labelfont=bf,labelsep=period,justification=raggedright,singlelinecheck=off]{caption}
\renewcommand{\figurename}{Fig}

% Use the PLoS provided BiBTeX style
% \bibliographystyle{plos2015}

% Remove brackets from numbering in List of References
\makeatletter
\renewcommand{\@biblabel}[1]{\quad#1.}
\makeatother



% Header and Footer with logo
\usepackage{lastpage,fancyhdr,graphicx}
\usepackage{epstopdf}
%\pagestyle{myheadings}
\pagestyle{fancy}
\fancyhf{}
%\setlength{\headheight}{27.023pt}
%\lhead{\includegraphics[width=2.0in]{PLOS-submission.eps}}
\rfoot{\thepage/\pageref{LastPage}}
\renewcommand{\headrulewidth}{0pt}
\renewcommand{\footrule}{\hrule height 2pt \vspace{2mm}}
\fancyheadoffset[L]{2.25in}
\fancyfootoffset[L]{2.25in}
\lfoot{\today}

%% Include all macros below

\newcommand{\lorem}{{\bf LOREM}}
\newcommand{\ipsum}{{\bf IPSUM}}



\usepackage{lineno}
\usepackage{setspace}\doublespacing



\usepackage{forarray}
\usepackage{xstring}
\newcommand{\getIndex}[2]{
  \ForEach{,}{\IfEq{#1}{\thislevelitem}{\number\thislevelcount\ExitForEach}{}}{#2}
}

\setcounter{secnumdepth}{0}

\newcommand{\getAff}[1]{
  \getIndex{#1}{MPI,UPENN,NMSA,Leiden,Tube,UOW}
}

\providecommand{\tightlist}{%
  \setlength{\itemsep}{0pt}\setlength{\parskip}{0pt}}

\begin{document}
\vspace*{0.2in}

% Title must be 250 characters or less.
\begin{flushleft}
{\Large
\textbf\newline{Introducing Platform Surface Interior Angle and Its Role in Flake
Formation, Size and Shape} % Please use "sentence case" for title and headings (capitalize only the first word in a title (or heading), the first word in a subtitle (or subheading), and any proper nouns).
}
\newline
% Insert author names, affiliations and corresponding author email (do not include titles, positions, or degrees).
\\
Shannon P. McPherron\textsuperscript{\getAff{MPI}}\textsuperscript{*},
Aylar Abdolahzadeh\textsuperscript{\getAff{UPENN}},
Will Archer\textsuperscript{\getAff{NMSA}},
Igor Djakovic\textsuperscript{\getAff{Leiden}},
Tamara Dogandžić\textsuperscript{\getAff{MPI}},
Li Li\textsuperscript{\getAff{Tube}},
Sam Lin\textsuperscript{\getAff{UOW}},
Jonathan Reeves\textsuperscript{\getAff{Tube}},
Zeljko Rezek\textsuperscript{\getAff{MPI}},
Marcel Weiss\textsuperscript{\getAff{MPI}}\\
\bigskip
\textbf{\getAff{MPI}}Department of Human Evolution, Max Planck Institute for Evolutionary
Anthropology, Leipzig, Germany\\
\textbf{\getAff{UPENN}}Department of Anthropology, University of Pennsylvania, Philadelphia,
PA, USA\\
\textbf{\getAff{NMSA}}The National Museum of South Africa, Bloemfontein, South Africa\\
\textbf{\getAff{Leiden}}Department of Archaeology, University of Leiden, Leiden, The Netherlands\\
\textbf{\getAff{Tube}}Department of Early Prehistory and Quaternary Ecology, Eberhard Karls
University of Tübingen, Tübingen, Germany\\
\textbf{\getAff{UOW}}Centre for Archaeological Science, School of Earth, Atmospheric and Life
Sciences, University of Wollongong, Wollongong, Australia\\
\bigskip
* Corresponding author: mcpherron@eva.mpg.de\\
\end{flushleft}
% Please keep the abstract below 300 words
\section*{Abstract}
Four ways archaeologists have tried to gain insights into how
flintknapping creates lithic variability are fracture mechanics,
controlled experimentation, replication and attribute studies of lithic
assemblages. Fracture mechanics has the advantage of drawing more
directly on first principles derived from physics and material sciences,
but its relevance to controlled experimentation, replication and lithic
studies more generally has been limited. Controlled experiments have the
advantage of being able to isolate and quantify the contribution of
individual variables to knapping outcomes, and the results of these
experiments have provided models of flake formation that when applied to
the archaeological record of flintknapping have provided insights into
past behavior. Here we develop a linkage between fracture mechanics and
the results of previous controlled experiments to increase their
combined explanatory and predictive power. We do this by documenting the
influence of Herztian cone formation, a constant in fracture mechanics,
on flake platforms. We find that the platform width is a function of the
Hertzian cone constant angle and the geometry of the platform edge. This
finding strengthens the foundation of one of the more influential models
emerging from the controlled experiments. With additional work, this
should make it possible to merge more of the experimental results into a
more comprehensive model of flake formation.

% Please keep the Author Summary between 150 and 200 words
% Use first person. PLOS ONE authors please skip this step.
% Author Summary not valid for PLOS ONE submissions.

\linenumbers

% Use "Eq" instead of "Equation" for equation citations.
\hypertarget{introduction}{%
\section{Introduction}\label{introduction}}

There is considerable literature dedicated to understanding how flakes
form. This understanding has come from work following several broad
approaches including fracture mechanics (e.g.~{[}1--4{]}), controlled
experiments (e.g.~{[}5--13{]}), replicative experiments (e.g.~Eren
etc.), and attribute analysis of archaeological assemblages. These
approaches each have their own strengths and weaknesses; however, one
way to think about the differences between them is in the directionality
of inference. Fracture mechanics starts with first principles, or laws
drawn from physics and material sciences in particular, concerning how
fractures should form in brittle solids. These are then used to make
predictions, typically with explicit mathematical equations, about how
flakes should look (size and shape) under varying conditions of force
application and solid material properties. These predictions are are
then typically tested experimentally. To the contrary, controlled and
replicative experiments and studies of actual lithic assemblages look at
empirical regularities in size and shape under varying conditions of
core preparation and flaking (where the core is struck, the hammer type,
how the platform is prepared, the angle of strike, etc.) to build
statistical models of flake formation. From these one can try to infer
first principles, but for reasons discussed next, this is typically not
done. All of these approaches to understanding flake formation are, of
course, valid and useful. Ideally what is learned from actually doing
(experiments) and what is learned from knowing how it should work in
principle (fracture mechanics) inform each other in an iterative loop.

Understanding first principles causality from statistical modeling,
however, is challenging. McElreath {[}14{]} gives the example of trying
to understand the physics behind race cars by measuring their
attributes. Knowing the speed and handling characteristics of each car,
eventually the right things could be measured on each car to build
statistical models with enough predictive power to know how a new car
might perform in a race, but it would be quite difficult to infer the
general physical concepts (and laws) like torque, angular momentum,
friction, conservation of energy, etc. from these statistical models. Of
course, with prior knowledge of the physics, finding the right
attributes to measure on the cars and statistical modeling are more
quickly and accurately done. This is important because even when the
physical laws are known, modeling them directly can be prohibitively
complex or computationally expensive (e.g.~air resistance) whereas
experiments and statistical models can more efficiently arrive at
useable solutions.

The same is true of studies of flake formation and the role of fracture
mechanics within it. Fracture mechanics itself is a longstanding and
widely-applied field of study, but its practical application has been
extremely limited in our field, with the best examples coming from the
papers of Cotterell and Kamminga {[}2,3{]} and of Speth {[}4{]}. These
papers start with the physics of fracture mechanics in brittle solids to
then explain how flakes are formed and, therefore, why they vary. Some
attributes, like the bulb of percussion, are more easily directly
accounted for in fracture mechanics (e.g.~Hertzian cone formation),
whereas for other attributes, like flake size and shape, the conceptual
and mathematical frameworks are provided by fracture mechanics. However,
as with the just mentioned example of air resistance, translating the
physics of how flakes are formed into a workable model that can predict
flake size and shape given the relevant parameters (e.g.~core shape,
angle of blow, force of blow, etc.) has not come to pass (cf.~{[}4{]}),
and it may not come to pass any time soon.

So instead, while some papers on controlled experiments in flake
formation may cite papers from fracture mechanics, their approaches are
all based on statistical modeling of the relationships between variables
of flaking and flake outcomes. Speth's work on this topic is a good
example. His 1972 paper uses fracture mechanics to derive a formula to
predict flake size which is then tested against a set of actual flakes
from a prehistoric site. By 1975 and again in 1981, Speth had moved to
experimental approaches (ball bearings on glass) and the connection back
to fracture mechanics had all but disappeared. Dibble {[}10{]} goes
further and dismisses fracture mechanics from the start as nearly
irrelevant. Instead of looking to fracture mechanics for insights into
what to study, experimental studies are being informed by replicative
knappers and observations on how actual lithic assemblages vary. Dibble
{[}10,15{]} is explicit in stating that his experimental research is
based on what knappers would have been able to control. Coming back to
the race car analogy, we are carefully building cars controlling for
engine size, wheel configurations, foils, etc., things that are
generally thought to be important for making a car go fast, and then
measuring their speeds. Again, though, because it is also difficult to
go in the other direction (from statistical modeling back to first
principles), the controlled experiment papers have not produced a
general model of how flakes form. Instead, we have a series of
statistical models that are difficult to relate to one another
(e.g.~{[}16{]}). The strongest and most influential of these is the
exterior platform angle and platform depth (EPA-PD) model.

The EPA-PD model states that flake size (weight) is primarily a function
of two important variables: exterior platform angle (EPA) and platform
depth (PD). Increasing either increases flake size, but the relationship
between the two is geometric such that at higher values of EPA changes
in PD have a greater effect on flake size. The EPA-PD model has been
replicated in multiple ways, including experiments in the material
sciences {[}6--8,10,17,18{]} and in actual lithic assemblages
(e.g.~{[}15,19,20{]}). It is also argued that in certain conditions,
EPA-PD has a stronger effect on flake size and shape than does core
surface morphology {[}11,21{]}. The EPA-PD model of flake formation,
however, is constrained on what it can explain. For instance, beveled
flakes, where the volume behind the platform is thinned, are not easily
included into the model {[}22{]}. Beveled flakes are typically larger
(weight) than the EPA-PD models predicts given their lower platform
depths. The EPA-PD model also does not explain why flake size and shape
change with varying angles of blow and platform shape {[}9,16,23{]}. It
also does not account for flake width, which is obviously a major
component of shape. It is worth noting that while the percentage of
variability in weight explained by the EPA-PD model (R2) is typically
high in the Dibble glass experiments, it is far lower in actual lithic
assemblages {[}23{]}. It is low enough that its utility for measuring
retouch intensity (i.e.~knowing how much mass has been removed from a
flake through retouch) is limited {[}24,25{]}, and so there have been
various proposals to improve the statistical modeling of flake weight
(or size) from different sets of measures (e.g.~{[}24,26--28{]}). Again,
though, without a general model, it is not really clear why one
measurement technique should work better than another, and so success is
measured by R2 values rather than against a theoretical predictions. The
problem is that our knowledge of how flake formation works is still too
limited to be translated into measurable attributes.

Here we propose to build on the EPA-PD model by 1) switching the focus
from variables controlled by the knapper to variables that might be more
directly related to flake initiation and formation and by 2) drawing
insights from the fracture mechanics literature. In particular, we start
with the proposal that the Hertzian cone, the angle of which is know
from fracture mechanics to be a constant for a given raw material, has a
measurable impact on flake formation.

When a core is struck, the force of impact begins to produce a Hertzian
cone or ``cone crack'' at the point of percussion (Figure?). The
Hertzian cone is the characteristic feature of Hertzian fracture and is
produced when a hard indenter is pressed onto the flat surface of a
brittle solid {[}29--33{]}. In three dimensions it forms a truncated
cone below the impact surface (see figure). In archaeology, Hertzian
fracture is often referred to as conchoidal fracture. While the size of
the Hertzian cone is dependent on variables such as the indenter's
radius, impact velocity and fracture toughness of the brittle solid, the
cone angle remains unchanged {[}34{]}. The Hertzian cone angle
(sometimes referred to as the included angle in fracture mechanics
literature) is the apex angle of the Hertzian cone. Though the
orientation of the Hertzian cone can be altered by the changing the of
angle of blow {[}32,34--36{]}, the apex angle remains constant for a
given raw material type {[}37{]}. A number of fracture mechanics studies
have demonstrated that the Hertzian cone angle is approximately 136° for
soda lime glass {[}31--33{]}. This value is also reported in Cotterell
and Kamminga {[}2{]}.

The cone crack grows as force is transferred to the core. Once the crack
reaches a certain length, the crack's propagation path will no longer be
Hertzian. At this point, instead of propagating along edges of the
initial Hertzian cone, the crack continues almost parallel to the core
surface to form a flake {[}1,2{]}. Although the Hertzian cone is only
associated with the initial crack formation, the angle of the cone
leaves a marked effect on the ventral surface of the flake near where it
has been struck (Figure ?). Our prediction is that the constant Hertzian
cone angle means that the platform width is largely a function of where
the core is struck and where this angle interests the platform edge. To
show this we present a new measure called platform surface interior
angle (PSIA) formed by the point of percussion and the extent of the
platform width. We predict that this new angle will be constant given
that if directly follows from the constant of the Hertzian cone angle.
If our prediction is correct, platform width (PW) is integrated into the
EPA-PD model and grounded in fracture mechanics via PSIA. This model
also has behavioral implications in that it may explain how the
manipulation of the platform impacts flake morphology.

To test this model, we examine several sets of flakes, including flakes
produced in the Dibble glass experiments and flakes from replication
experiments, using several methods to measure the PSIA. We find that the
mean angle in all datasets, regardless of how it is measured, is the
same (approximately 136 degrees) and quite consistent with above
mentioned values for Hertzian cone formation {[}2{]}. There is some
variability in the PSIA, and it is clear that this variability cannot be
solely attributed to measurement error. In the Dibble glass experiments,
where key variables are controlled, there is some indication that the
PSIA responds to the angle of blow. Our finding is consistent with all
of the empirical results of the Dibble experiments and it explains some
of the patterns in those data that previously were unaccounted for. Once
combine PSIA with the existing EPA-PD model, we may have a model for
flake formation that can explain a larger portion of the variability we
see in stone tool assemblages and that may allow for a closer link to
predictions coming from fracture mechanics.

\hypertarget{materials-and-methods}{%
\section{Materials and Methods}\label{materials-and-methods}}

We examine the platform surface interior angle in three different
datasets. First, we examine glass flakes and cores coming from the
Dibble controlled experiments in flake formation {[}9,16,19,21,22{]}.
This dataset has the advantage that a number of potentially important
variables are either controlled for or measurable. These include the
exterior platform angle, the angle of blow, the hammer type, raw
material, and metrics such as platform thickness, platform width, flake
length, width and thickness, and flake weight. Hereafter this dataset is
referred to as the \emph{Dibble glass data}. Second, we attempt to
replicate the findings from the Dibble glass data by measuring the PSIA
in a large (n = 568) set of flakes complete, unretouched flakes coming
from 45 discrete reduction sequences produced in replicative experiments
by three knappers who were naïve to the goals of this study. These
flakes were made with the intent of replicating various Middle and Upper
Paleolithic core reduction strategies from the initial decortification
of the core through to flake production and core maintenance. These
replicative experiments used nodules of high-quality Bergerac and
Sénonien weight from \textasciitilde480--4100 g and coming from the
southwest region of France {[}38{]}. For each of the flakes coming from
these reduction series, the technology and the type of hammer (hard
hammer, soft hammer and indirect percussion) are known. Today these
flakes are stored in Campagne, France, and hereafter this dataset is
referred to as the \emph{Campagne data} (see {[}39{]} for additional
details on the structure of this dataset). Third, in addition, we
measured a small set of flakes produced at the Max Planck Institute for
Evolutionary Anthropology in Leipzig, Germany, in the context of
teaching, replication, and experimentation. In this case, no details are
known about how the flakes were produced, and this set of flakes is used
here only to test a method for measuring the PSIA. Hereafter this
dataset is referred to as the \emph{MPI data}.

The methods used to measure the PSIA varied substantially between the
three datasets for practical reasons given the types of data available
to us and to begin to have some idea of how best to measure this angle
in future studies. First, for the Dibble glass data, we used the
following procedure. Dibble and colleagues used several core forms, but
the first and most common type is what was called the semispherical
core. This core (reproduced here) looks like a loaf of bread with flat,
squared off sides and back, and a curved or domed flaking surface. An
unworked example of this core type was scanned using an NextEngine 3D
laser scanner. The resulting mesh was then processed in R to rotate the
platform to be perpendicular to the Z axis (or coincident with the XY
plane). The XY coordinates of the triangles forming the platform were
then extracted from this model and a convex hull fit to this cloud of
points to have the full outline of the platform on the Dibble
semispherical cores (Figure 1). Next, we extracted just the portion of
the outline that is where flakes are struck from these cores, and we fit
a polynomial curve to these points. Using the formula for this curve, we
created a series of equally spaced (in X) points along the platform edge
(see Figure 1). We then filtered the Dibble glass data to have only
flakes made from the semispherical cores by hard hammer. We include only
flakes with a feather termination, and we exclude flakes coming from
experiments on platform beveling {[}22{]} and so-called `on-edge' core
strikes {[}16{]}. Knowing that Dibble and colleagues tried to strike
flakes from these cores at the center or peak of the core surface
curvature, we use the platform depths reported for these flakes to
position the point of percussion relative to the set of platform edge
outline points described above. Next, we find the symmetrical pair of
platform edge outline points, one to the left and one to the right of
the point of percussion, that yield a platform width equal to the
reported platform width for each flake. Finally, the PSIA is calculated
(arc-cosine of the dot product of two normalize vectors) as the angle
between the two line segments formed by the left platform width point
and the point of percussion and the right platform point and the point
of percussion. In the results presented below, this angle is referred to
as the \emph{estimated} PSIA to indicate that is is not directly
measured from the flakes themselves.

{[}I will insert a new figure showing a Dibble glass core and how the
measures we talk about here are made{]}

\begin{figure}
\centering
\includegraphics{PSIA_Manuscript_files/figure-latex/fig1-core_outlines-1.pdf}
\caption{Outline of Dibble core surface (left) and the polynominal
fitted to the core edge (right).}
\end{figure}

We note that there are several potential sources of error in this
method. First, we are assuming that each flake is struck from the center
of the core, and while this was the intention in the glass experiments,
there is certainly some error associated with this. Second, we are
assuming that the flake fracture plane is parallel to the core surface
and not twisted towards one lateral side or the other. To the extent
that either of these assumptions is invalid, it will impact the angle
calculation.

To verify the angles computed in this way from the Dibble glass data, we
also measure this angle directly with digital calipers and a goniometer
on a subset of these flakes. We use two methods of measurement to begin
to test how best to measure the PSIA by hand. In the first method, we
measure the three sides of the triangle formed by the two platform width
points and the point of percussion using digital calipers precise to .01
mm. Using standard trigonometric formulas, we then calculate the
interior angle of this triangle that corresponds to the PSIA as
described above. In the second method, we use a digital goniometer
precise to 0.1 degrees to record this angle. The joint of the goniometer
is positioned at the point of percussion and the jaws positioned to
cross the two platform width points. Both of these methods come with
possibilities for error. Both are impacted by one's ability to pinpoint
the point of percussion. In the Dibble glass flakes, because the core
edge is standardized, identifying the two platform width points is
fairly straightforward. However, in the goniometer method, taking the
measurement to these points while avoiding the curvature of the bulb of
percussion is not without some difficulties.

For the Campagne data, we use the following procedure. All of the flakes
were scanned using an Artec surface scanner. Each of the flake meshes
was then landmarked (see {[}39{]} for additional details on the scanning
and landmarking). For our purposes, three of these landmarks are
important: the two points (left and right) where the interior platform
intersects the core surface (i.e., the two ends of the platform width)
and the point of percussion. These three points are analogous with the
three points described above for computing the PSIA. This angle,
therefore, can be once again computed using the dot product of these two
line segments (specifically the arc-cosine of the dot product of the
normalize line segments). However, there is an important difference in
that with the Dibble glass data all computations are with two
dimensional line segments and in the Campagne dataset the line segments
are in three dimensions. In the latter case, the angle is computed in a
two dimensional plane that is coincident with both line segments, but we
note this difference because it could introduce a certain amount of
incomparability in the two datasets. Our expectation is that these
angles could average larger than the Dibble glass data because, for
instance, lifting the point of percussion relative to the two platform
points would result in a larger platform surface interior angle.

Lastly, for the MPI data, we use only the goniometer method described
above. One of us (MW) made the measurements with instructions only on
the mechanics of the measurement. To avoid bias, MW was naïve to the
goals and results of this study. In the course of measuring the flakes,
several problematic platforms were identified where the measurement of
the PSIA was not as clear as the person selecting the flakes (SPM) had
initially believed. These flakes were removed from the analysis.

We use the R {[}40{]} statistical environment to do this analysis. This
paper is an rMarkdown document, and it is included in the supplementary
information along with the data files needed to compile the document and
replicate all of the figures, tables, and statistics.

\hypertarget{results}{%
\section{Results}\label{results}}

\begin{figure}
\centering
\includegraphics{PSIA_Manuscript_files/figure-latex/fig2-test_1_angles-1.pdf}
\caption{Distribution of estimated PSIA based on the Dibble glass
flakes.}
\end{figure}

Figure 2 shows the distribution of estimated PSIA in the Dibble glass
dataset. The distribution has a mean of 136.49±7.56. Variation in this
angle does not seem to be related to platform depth, exterior platform
angle or weight (Figure 3). There is a relationship between platform
width and the platform surface interior angle such that larger angles
result in wider platforms, which is to be expected (see Figure 3).

\begin{figure}
\centering
\includegraphics{PSIA_Manuscript_files/figure-latex/fig3-angles_to_other_measures-1.pdf}
\caption{PW, PD, EPA and weight against the estimated PSIA based on the
Dibble glass flakes.}
\end{figure}

There also appears to be a relationship between the angle of blow and
the PSIA (Figure 4). While there are fewer cases with angles of blow
less than 20, there is some indication in the data that lower angles of
blow may be correlated with lower PSIA. However, once an angle of blow
is between 10-20 degrees or higher, the angle of blow is not correlated
with the PSIA.

\begin{figure}
\centering
\includegraphics{PSIA_Manuscript_files/figure-latex/fig4-AOB_to_PSIA-1.pdf}
\caption{AOB against the estimated platform surface interior angles
based on the Dibble glass flakes.}
\end{figure}

Another way of looking at the relationship between platform width and
PSIA is to calculate what the platform width would be if the PSIA is a
constant and compare this to the actual platform width. We can do this
by placing the point of percussion on the same platform outlines as
above using the known platform depth for each of the flakes in the
Dibble glass data set. We then use the average PSIA computed above to
extend two vectors from this point of percussion to the platform edge.
Where these vectors intersect the platform edge defines the left and
right limits of the platform width. This estimated value is then plotted
against the actual, measured platform widths (Figure 5).

\begin{figure}
\centering
\includegraphics{PSIA_Manuscript_files/figure-latex/fig5-pd_pw_with_estimated_pw-1.pdf}
\caption{The actual platform depth to platform width data from the
Dibble glass core flakes and the estimated platform width using the
average platform surface interior angle calculated previously.}
\end{figure}

Figures 6 and 7 show comparisons of the results of the estimated PSIA
presented above with direct measurements of this angle on a sample of 49
of the Dibble glass flakes. For this sample, the PSIA is 135.71±4.86.
When measured with a digital goniometer the angle is 133.44±4.61, and
when measured with digital calipers and calculated using trigonometry
the angle is 135.86±8.85.

\begin{figure}
\centering
\includegraphics{PSIA_Manuscript_files/figure-latex/fig6_remeasure_comparisons-1.pdf}
\caption{A comparison of the estimated platform surface interior angle,
this same angle as calculated from caliper measurements, and this same
angle measured with a goniometer. All measures are in degrees.}
\end{figure}

\begin{figure}
\centering
\includegraphics{PSIA_Manuscript_files/figure-latex/fig7_remeasure_distributions-1.pdf}
\caption{The distributions of platform surface interior angle as
directly measured with a goniometer and as calculated from caliper
measurements.}
\end{figure}

The distribution of the PSIA for the 568 flakes in the Campagne dataset
is shown in Figure 8. In the hard hammer flakes only, which is the
technique used in our subset of the Dibble glass dataset, the mean
platform surface interior angle is 140.36±12.38. Because data are
available for punch and soft hammer flakes, these are presented as well.
Punch flakes have a lower PSIA (126.53±11.01), and soft hammer flakes
have a mean of 138.94±15.69. The mean PSIA for all flakes in the
Campagne data set is 138.63±13.27.

\begin{figure}
\centering
\includegraphics{PSIA_Manuscript_files/figure-latex/fig8-camp_angles-1.pdf}
\caption{Distribution of PSIA in the Campagne data set color coded by
percussion type.}
\end{figure}

In the Campagne data, PSIA does not covary with platform width, platform
depth or the shape of the platform (as measured by the ratio of platform
width to platform depth). Though sample size is potentially a problem,
there is perhaps an indication that for larger platform depths, there is
less variability in the PSIA.

\begin{figure}
\centering
\includegraphics{PSIA_Manuscript_files/figure-latex/fig9-PSIA_to_other_measures-1.pdf}
\caption{PSIA as a function of platform width, platform depth, and EPA
in the experimental flake collection. Color coding is the same as in
Figure 8.}
\end{figure}

The distribution of PSIA in the MPI dataset as measured by a digital
goniometer is presented in Figure 10. This dataset contains \texttt{67}
flakes, and the mean is 137.75±10.97, in keeping with the other
datasets.

\begin{figure}
\centering
\includegraphics{PSIA_Manuscript_files/figure-latex/fig10-mpi_data-1.pdf}
\caption{Distribution of platform surface interior angle in the MPI
flakes dataset as measured by a digital goniometer.}
\end{figure}

\hypertarget{discussion}{%
\section{Discussion}\label{discussion}}

Each of the datasets and measurement methods used to determine the PSIA
yielded very similar results, and these results are consistent with the
prediction based on the already known and constant angle of Hertzian
cones (136-137 degrees). While these results directly incorporate
platform width into the EPA-PD model of flake formation, they also
suggest that the existing model requires additional nuance. All other
things being equal, the relationship between PSIA and platform width
indicates that what determines the size of the flake is how far into the
core it is struck. Specifically, because of the constant PSIA, the
deeper into the core the flake is struck, the greater the platform width
will become (and the thicker the flake will become as well). However,
and this is the nuance, while platform depth is normally a good proxy
for how far into the core the flake is struck, it is not exactly the
same thing. What platform depth really measures is how far into the
\emph{existing platform} of the core the flake is struck. Beveled flakes
illustrate this point.

Beveled flakes are ones where material is removed behind the platform
prior to striking the core. Dibble and colleagues recognized that
beveling altered the EPA-PD model of flake formation such that the
interaction of platform depth and exterior platform angle no longer
predicted flake size {[}22{]}. Beveled flakes have too thin a platform
for their size. However, PSIA helps explain this discrepancy and thereby
pulls pulls beveled flakes back into an expanded EPA-PD model. Too
illustrate this point, we examine some beveled flakes in the Dibble
glass dataset.

In the Dibble glass dataset there are 11 flakes with flat or concave
beveling, coming from semispherical cores, and otherwise conforming to
our selection criteria. They are plotted in Figure 11 along with the
non-beveled flakes. It is clear that beveling changes the relationship
between platform depth and platform width. For a given platform width,
the beveled flakes have much shallower platforms (smaller PD) than
expected. As a result, the EPA-PD model underestimates their weight (see
Figure 11) as well. To illustrate the power of PSIA to model these
flakes, we build a linear model to estimate platform depth from the PSIA
and platform width in the non-beveled flakes in the Dibble glass
dataset. We then use this model to predict a platform depth for the
beveled flakes. However, given that the linear model requires PSIA to
predict platform depth and the actual PSIA is not known for these
flakes, we substitute the average PSIA, as reported above for the
non-beveled flakes, in its place. When this is done, the predicted
platform depths for the beveled flakes plot on the same trend line as
the non-beveled flakes (compare red and blue points in Figure 11).

This modeled platform depth is then be used to improve the EPA-PD model
to give better estimates of flake size. The main aspect of size that
Dibble and colleagues have focused on with the EPA-PD model is weight,
and so we model flake weight as a function of EPA, platform depth and
the interaction of the two (see Figure 11). The cube root of weight is
used to correct for the different dimensions in the model. Next, we use
this same model to predict flake weight in the beveled flakes using the
platform depth as originally measured on these beveled flakes. In this
case, the modeled flake weights are much too low in comparison to their
actual weights (red points in Figure 11). Finally, we use the predicted
platform depth for the beveled flakes, as modeled above, to predict
flake weight again using the non-beveled flake model. In this case, the
flake weights plot in among the rest of the non-beveled flakes (blue
points in Figure 11). Thus the beveled flakes are the expected size when
we think of PD in the EPA-PD model not as a measure of platform depth
but rather as a measure of how far into the core the flake is struck,
which then determines the flake width via the PSIA given the shape of
the platform edge. Beveling does not change the expected size of these
flakes when flake formation is viewed this way.

\begin{figure}
\centering
\includegraphics{PSIA_Manuscript_files/figure-latex/fig11-bevel_with_estimated_pd-1.pdf}
\caption{Platform depth to platform width including beveled flakes (red)
and non-beveled flakes (black) (top). Estimated points in blue are the
same beveled flakes but with the platform depth predicted using the
average PSIA and their actual platform width. At the bottom, the actual
flake weight is compared to the predicted flake weight based on an
EPA-PD model for non-beveled flakes (points in black). The predicted
weight using the actual (w/o PSIA) and the modeled (with PSIA) platform
depths for the beveled flakes are then plotted as well in red and blue
respectively.}
\end{figure}

There is some indication in the Dibble glass data that the angle of blow
may impact the PSIA. At low angles of blow (with 0 degrees representing
an angle of blow perpendicular to the platform, see Figure 2 in
{[}16{]}), the PSIA is below average, and it appears to increase until
the angle of blow reaches between 10 and 20 degrees (from perpendicular)
after which the PSIA remains essentially unchanged (see Figure ). With
the caveat that the Dibble glass data set has very few cases with angles
of less than 20 degrees, the fracture mechanics literature suggests a
relationship like this: the angle of blow changes the direction, though
not the size, of the Hertzian cone such that flakes struck from cores
with a high angle of blow (oblique strike) should have ``steeper and
less prominent cones and less salient bulbs of percussion than flakes
which are struck more steeply {[}or more perpendicular{]}'' ({[}4{]},
page 38). Experimentally, however, Magnani et al.~{[}16{]} seem to find
the opposite. They report that a negative angle of blow (here values
less than 0 meaning a strike directed into the interior of the core
rather than towards the core surface) have smaller bulbs relative to the
weight of the flake. In our Campagne data set, we see a difference
between flakes made from direct hard-hammer percussion and those made
with a punch technique. The latter cluster at the low range of PSIA.
This difference could be interpreted as reflecting a difference in the
angle of blow in that punch flakes are more likely to be struck
perpendicularly to the platform surface (an angle of blow of 0). More
work needs to be done, in particular in analyzing the Dibble glass data
set where angle of blow is well controlled, but we suggest that
increasing the angle of blow has the effect of tipping the direction of
the Hertzian cone such that it intersects the core surface not as a
circle but rather as an ellipse. This is a phenomenon that has been
repeatedly documented in fracture mechanics {[}32,34,36{]}. Although the
angle of the cone itself remains unchanged, its intersection with the
surface broadens and results in higher PSIA. Thus, if this suggestion is
correct, striking a core with a high angle of blow will result in a
larger platform width for a given platform thickness.

The direct measurement of PSIA in a subsample of the Dibble glass flakes
shows that our method for finding this angle using the platform surface
shape and platform depth is working. However, there is variability in
this angle depending on how it is measured. With the caveats that only
one person measured these flakes and that the sample is small, in
general it seems that the direct measurement with a goniometer performs
better than the indirect calculation of the angle from three caliper
measurements. Of the two methods, the caliper measurements show greater
variability than do the goniometer measurements. The caliper method can
also fail completely when measurement error produces a triangle with
impossible side lengths (e.g.~the sum of the two shorter sides does not
equal the length of the longest side). Importantly too, this is only
knowable once the measures are taken and an angle is calculated, making
it much more difficult to correct, whereas the goniometer method
produces an angle each time. Our study, however, does not indicate which
of the three methods in this case is correct, nor do we know the error
associated with any of these individual methods. Now that PSIA seems to
produce results relevant to understanding flake formation, additional
studies are required to better know the error distribution on an
individual measure. We note too that this error distribution will likely
vary with the angle itself, the size of the point of percussion, and
other factors that remain to be identified. The question, however, is
whether this measurement error will overwhelm patterns, such as, for
instance, showing a potential correlation in angle of blow and PSIA. It
is our expectation that with a digital goniometer, direct measurement of
the PSIA can become one of the standard measurements within lithic
attribute analysis, but this remains to be determined.

We note that our finding that the platform surface interior angle varies
around a constant derived from fracture mechanics appears to be
consistent with all of the findings to date of the Dibble glass
experiments {[}9,16,19,21,22{]}. Additionally, it perhaps helps explain
or account for one of the more counter intuitive findings of the glass
experiments, namely that flake size (weight) is not impacted by the
force with which the core is struck {[}9{]}. In the EPA-PD model, the
amount of force required to remove a flake given a particular
combination of EPA-PD is a constant (see also {[}17{]}). Subtracting
force means that the flake is not initiated. Adding force does not
change the size of the resulting flake. This makes sense in the PSIA
addition to the model. In fracture mechanics, it is known that striking
a material harder will change the size of the Hertzian cone but not the
angle {[}31{]}. Thus when a given core is struck, how far into the core
the Hertzian cone can travel will be depend on the hammer size and the
striking force, but where the cone will intersect the core surface does
not change. So striking a Dibble glass core harder at a particular point
does not change the platform width, and as a result, the subsequent
fracture plane that removes the flake has much less freedom to change
the size of the flake. This said, we note that a recent study {[}36{]}
shows that increasing impact velocity of the indenter will eventually
cause the Hertzian cone angle to decrease (see also {[}41{]}). The
question is whether these velocities are relevant to stone knapping.

\hypertarget{conclusions}{%
\section{Conclusions}\label{conclusions}}

Fracture mechanics is a massive field of study with both great potential
and great difficulties for understanding flake formation. The potential
is that the physical laws and models coming from fracture mechanics are,
ultimately, how the actions of knappers are translated to usable flakes.
The difficulties are both inherent to the field of fracture mechanics
itself and to the complexity of the problem (some solutions require more
time and computing power than exists), and the difficulties relate to
the challenges of interdisciplinary work where the equations and goals
of one field are terribly difficult to bring into another. This later
point is clearly seen in the early fracture mechanics literature and the
minimal impact it has had on experimental and replicative studies of
flake formation.

This said, our goal here was to return to this literature and to try to
find some useful insights that could be translated to a better
understanding of the underlying mechanisms (or first principles) of
flake formation and that might thereby lead to a better integration of
the various statistical models currently in use. To do this, we
discarded the existing focus in the experimental literature on what
knappers do and instead focused on attributes that may be more directly
related to fracture mechanics. Thus we focused our attention on the
Hertzian percussion cone as a constant and wondered if it could help
explain platform width, an aspect of flake size and shape that up to now
has been absent from the dominant EPA-PD model of flake formation. We
measured three different collections in multiple different ways and
found that in each case the angle formed by the platform width and the
point of percussion to be, on average, essentially the same as what is
predicted from fracture mechanics for the angle of the Hertzian cone.

We conclude that the platform surface interior angle or PSIA is an
important determinant in flake size and shape. While it would seem that
it is a constant and not under direct control by the knapper, the
knapper is (unknowingly) exploiting the properties of this angle when
preparing the platform and its contact with the core surface and when
deciding how far into the core to strike. In our model, this is how the
PD side of the EPA-PD model of flake formation is translated into a
flake of a particular size and shape. In other words, it is not the PD
that directly structures flake size but rather PD is proxy for how far
into the core a flake is struck which then, through the effect of
platform surface interior angle on platform width, structures the size
and shape of the flake.

This, however, requires further testing, particularly with beveled
flakes where PD performs poorly as a proxy for how far into the core a
flake is struck. If the PSIA performs better in these circumstances,
then we will have improved the EPA-PD model and helped to integrate what
are now disparate studies {[}9,22{]}. Additionally, while our study
shows that the average PSIA conforms well to predictions from fracture
mechanics for Hertzian flake formation, it is still clear that there is
variability around this mean. Given that there is certainly some chaos
in flake formation, we are not sure how much variability to expect. For
instance, sometimes the Hertzian cone crack may kink at a light angle to
continue expanding, and this kink may contribute to the variability
observed in PSIA {[}42,43{]}. However, it is also clear that the model
does not work at all for some flakes. It may be the case that these
flakes are not formed by Hertzian mechanisms (e.g.~bending flakes) and
that alternative models will be required in these cases. Clearly more
experiments and more data are required to begin to understand which
kinds of flakes fail the PSIA model presented here, and these flakes
will require additional insights into models of flake formation.

\hypertarget{acknowledgements}{%
\section{Acknowledgements}\label{acknowledgements}}

Michel Brenet and Laurence Bourguignon produced the Campagne data set,
and we thank them for allowing us access to this important collection.
We thank the Max Planck Society for funding portions of this work. SPM,
MW, WA, ZR and TD thank Jean-Jacques Hublin for his support this
research agenda. Sadly Harold Dibble died before the main findings of
this paper were discovered. Without his vision and his efforts to build
a data set of flakes made under controlled conditions, this paper would
not have been possible. We dedicate this paper to him.

\hypertarget{references}{%
\section*{References}\label{references}}
\addcontentsline{toc}{section}{References}

\hypertarget{refs}{}
\leavevmode\hypertarget{ref-cotterell_essential_1985}{}%
1. Cotterell B, Kamminga J, Dickson FP. The essential mechanics of
conchoidal flaking. International Journal of Fracture. 1985;29:
205--221.

\leavevmode\hypertarget{ref-cotterell_formation_1987}{}%
2. Cotterell B, Kamminga J. The formation of flakes. American Antiquity.
1987; 675--708.

\leavevmode\hypertarget{ref-cotterell_mechanics_1979}{}%
3. Cotterell B, Kamminga J. The mechanics of flaking. In (B. Hayden, Ed)
Lithic Usewear Analysis. 1979;

\leavevmode\hypertarget{ref-speth_mechanical_1972}{}%
4. Speth JD. Mechanical basis of percussion flaking. American Antiquity.
1972; 34--60.

\leavevmode\hypertarget{ref-speth_experimental_1974}{}%
5. Speth JD. Experimental Investigations of Hard-Hammer Percussion
Flaking. 1974;

\leavevmode\hypertarget{ref-speth_miscellaneous_1975}{}%
6. Speth JD. Miscellaneous studies in hard-hammer percussion flaking:
The effects of oblique impact. American Antiquity. 1975;40: 203--207.

\leavevmode\hypertarget{ref-speth_role_1981}{}%
7. Speth JD. The role of platform angle and core size in hard-hammer
percussion flaking. Lithic Technology. 1981;10: 16--21.

\leavevmode\hypertarget{ref-dibble_effect_1995}{}%
8. Dibble HL, Pelcin A. The Effect of Hammer Mass and Velocity on Flake
Mass. Journal of Archaeological Science. 1995;22: 429--439.

\leavevmode\hypertarget{ref-dibble_introducing_2009-1}{}%
9. Dibble HL, Rezek Z. Introducing a new experimental design for
controlled studies of flake formation: Results for exterior platform
angle, platform depth, angle of blow, velocity, and force. Journal of
Archaeological Science. 2009;36: 1945--1954.

\leavevmode\hypertarget{ref-dibble_new_1981-1}{}%
10. Dibble HL, Whittaker JC. New experimental evidence on the relation
between percussion flaking and flake variation. Journal of
Archaeological Science. 1981;8: 283--296.

\leavevmode\hypertarget{ref-pelcin_effect_1997}{}%
11. Pelcin AW. The effect of core surface morphology on flake
attributes: Evidence from a controlled experiment. Journal of
Archaeological Science. 1997;24: 749--756.

\leavevmode\hypertarget{ref-pelcin_effect_1997-1}{}%
12. Pelcin A. The effect of indentor type on flake attributes: Evidence
from a controlled experiment. Journal of Archaeological Science.
1997;24: 613--621.

\leavevmode\hypertarget{ref-pelcin_formation_1997}{}%
13. Pelcin AW. The formation of flakes: The role of platform thickness
and exterior platform angle in the production of flake initiations and
terminations. Journal of Archaeological Science. 1997;24: 1107--1113.

\leavevmode\hypertarget{ref-mcelreath_sizing_2018}{}%
14. McElreath R. Sizing up human brain evolution. 2018;

\leavevmode\hypertarget{ref-dibble_platform_1997}{}%
15. Dibble HL. Platform variability and flake morphology: A comparison
of experimental and archaeological data and implications for
interpreting prehistoric lithic technological strategies. Lithic
technology. 1997;22: 150--170.

\leavevmode\hypertarget{ref-magnani_flake_2014-1}{}%
16. Magnani M, Rezek Z, Lin SC, Chan A, Dibble HL. Flake variation in
relation to the application of force. Journal of Archaeological Science.
2014;46: 37--49.

\leavevmode\hypertarget{ref-chai_edge_2007}{}%
17. Chai H, Lawn BR. Edge chipping of brittle materials: Effect of
side-wall inclination and loading angle. International Journal of
Fracture. 2007;145: 159--165.

\leavevmode\hypertarget{ref-dogandzic_results_2020}{}%
18. Dogandžić T, Abdolazadeh A, Leader G, Li L, McPherron SP, Tennie C,
et al. The results of lithic experiments performed on glass cores are
applicable to other raw materials. Archaeological and Anthropological
Sciences. 2020;12: 44.

\leavevmode\hypertarget{ref-lin_utility_2013-1}{}%
19. Lin SC, Rezek Z, Braun D, Dibble HL. On the utility and
economization of unretouched flakes: The effects of exterior platform
angle and platform depth. American Antiquity. 2013;78: 724--745.

\leavevmode\hypertarget{ref-rezek_two_2018-1}{}%
20. Režek Ž, Dibble HL, McPherron SP, Braun DR, Lin SC. Two million
years of flaking stone and the evolutionary efficiency of stone tool
technology. Nature ecology \& evolution. 2018;2: 628--633.

\leavevmode\hypertarget{ref-rezek_relative_2011-1}{}%
21. Rezek Z, Lin S, Iovita R, Dibble HL. The relative effects of core
surface morphology on flake shape and other attributes. Journal of
Archaeological Science. 2011;38: 1346--1359.

\leavevmode\hypertarget{ref-leader_effects_2017}{}%
22. Leader G, Abdolahzadeh A, Lin SC, Dibble HL. The effects of platform
beveling on flake variation. Journal of Archaeological Science: Reports.
2017;16: 213--223.

\leavevmode\hypertarget{ref-clarkson_estimating_2011-1}{}%
23. Clarkson C, Hiscock P. Estimating original flake mass from 3D scans
of platform area. Journal of Archaeological Science. 2011;38:
1062--1068.

\leavevmode\hypertarget{ref-dogandzic_edge_2015}{}%
24. Dogandžić T, Braun DR, McPherron SP. Edge Length and Surface Area of
a Blank: Experimental Assessment of Measures, Size Predictions and
Utility. PLOS ONE. 2015;10: e0133984.
doi:\href{https://doi.org/10.1371/journal.pone.0133984}{10.1371/journal.pone.0133984}

\leavevmode\hypertarget{ref-hiscock_generalization_2010}{}%
25. Hiscock P, Tabrett A. Generalization, inference and the
quantification of lithic reduction. World Archaeology. 2010;42:
545--561.

\leavevmode\hypertarget{ref-archer_geometric_2018}{}%
26. Archer W, Pop CM, Rezek Z, Schlager S, Lin SC, Weiss M, et al. A
geometric morphometric relationship predicts stone flake shape and size
variability. Archaeological and Anthropological Sciences. 2018;10:
1991--2003.

\leavevmode\hypertarget{ref-Braun2010}{}%
27. Braun DR, Rogers MJ, Harris JW, Walker SJ. Quantifying variation in
landscape-scale behaviors: The oldowan from koobi fora. In: Lycett S,
Chauhan P, editors. New perspectives on old stones: Analytical
approaches to paleolithic technologies. New York, NY: Springer New York;
2010. pp. 167--182.
doi:\href{https://doi.org/10.1007/978-1-4419-6861-6/\%0087}{10.1007/978-1-4419-6861-6\textbackslash\%0087}

\leavevmode\hypertarget{ref-muller_estimating_2014}{}%
28. Muller A, Clarkson C. Estimating original flake mass on blades using
3D platform area: Problems and prospects. Journal of Archaeological
Science. 2014;52: 31--38.

\leavevmode\hypertarget{ref-chen_contact_1995}{}%
29. Chen S, Farris T, Chandrasekari S. Contact mechanics of Hertzian
cone cracking. International journal of solids and structures. 1995;32:
329--340.

\leavevmode\hypertarget{ref-frank_theory_1967}{}%
30. Frank FC, Lawn B. On the theory of Hertzian fracture. Proceedings of
the Royal Society of London Series A Mathematical and Physical Sciences.
1967;299: 291--306.

\leavevmode\hypertarget{ref-kocer_angle_1998}{}%
31. Kocer C, Collins RE. Angle of Hertzian cone cracks. Journal of the
American ceramic Society. 1998;81: 1736--1742.

\leavevmode\hypertarget{ref-lawn_computer_1974}{}%
32. Lawn B, Wilshaw T, Hartley N. A computer simulation study of
Hertzian cone crack growth. International Journal of fracture. 1974;10:
1--16.

\leavevmode\hypertarget{ref-roesler_brittle_1956}{}%
33. Roesler F. Brittle fractures near equilibrium. Proceedings of the
Physical society section B. 1956;69: 981.

\leavevmode\hypertarget{ref-chen_influence_1989}{}%
34. Chen L-Y, Chaudhri MM. Influence of tangential stress on the impact
damage of soda-lime glass with spherical projecticles. 18th Intl
Congress on High Speed Photography and Photonics. 1989. p. 955.

\leavevmode\hypertarget{ref-akimune_oblique_1990}{}%
35. Akimune Y. Oblique impact of spherical particles onto silicon
nitride. Journal of the American Ceramic Society. 1990;73: 3607--3610.

\leavevmode\hypertarget{ref-chaudhri_dynamic_2015}{}%
36. Chaudhri MM. Dynamic fracture of inorganic glasses by hard spherical
and conical projectiles. Philosophical Transactions of the Royal Society
A: Mathematical, Physical and Engineering Sciences. 2015;373: 20140135.

\leavevmode\hypertarget{ref-fischer-cripps_introduction_2007}{}%
37. Fischer-Cripps AC. Introduction to contact mechanics. Springer;
2007.

\leavevmode\hypertarget{ref-jaubert_mousterien_2008}{}%
38. Turq A, Dibble H, Faivre J-P, Goldberg P, McPherron SJP, Sandgathe
D. Le Moustérien Récent Du Périgord Noir : Quoi De Neuf ? In: Jaubert J,
Bordes J-G, Ortega I, editors. Les Sociétés du Paléolithique dans un
Grand Sud-ouest de la France: Nouveaux gisements, nouveaux résultats,
nouvelles méthodes. Mémoire de la Société Préhistorique Française 48;
2008. pp. 83--94.

\leavevmode\hypertarget{ref-archer_quantifying_nodate}{}%
39. Archer W, Djakovic I, Brenet M, Bourguignon L, Presnyakova D,
Soressi M, et al. Quantifying differences in hominin flaking
technologies with 3D shape analysis.

\leavevmode\hypertarget{ref-team_r_2010}{}%
40. Team RDC. R: A Language and Environment for Statistical Computing.
Vienna, Austria; 2010.

\leavevmode\hypertarget{ref-wang_comparative_2017}{}%
41. Wang X-e, Yang J, Liu Q-f, Zhang Y-m, Zhao C. A comparative study of
numerical modelling techniques for the fracture of brittle materials
with specific reference to glass. Engineering Structures. Elsevier;
2017;152: 493--505.

\leavevmode\hypertarget{ref-hu_hertzian_1996}{}%
42. Hu S, Chen Z, Mecholsky JJ. On the Hertzian fatigue cone crack
propagation in ceramics. International journal of fracture. 1996;79:
295--307.

\leavevmode\hypertarget{ref-marimuthu_spherical_2016}{}%
43. Marimuthu KP, Rickhey F, Lee H, Lee JH. Spherical indentation
cracking in brittle materials: An XFEM study. 2016 7th International
Conference on Mechanical and Aerospace Engineering (ICMAE). IEEE; 2016.
pp. 267--273.

\nolinenumbers


\end{document}

